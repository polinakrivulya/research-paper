\documentclass{beamer}
\beamertemplatenavigationsymbolsempty
\usecolortheme{beaver}
\setbeamertemplate{blocks}[rounded=true, shadow=true]
\setbeamertemplate{footline}[page number]
%
\usepackage[utf8]{inputenc}
\usepackage[english, russian]{babel}
\usepackage{amssymb,amsfonts,amsmath,mathtext}
\usepackage{subfig}
\usepackage[all]{xy} % xy package for diagrams
\usepackage{array}
\usepackage{multicol}% many columns in slide
\usepackage{hyperref}% urls
\usepackage{hhline}%tables
% Your figures are here:
\graphicspath{ {fig/} {../fig/} }

%----------------------------------------------------------------------------------------------------------
\title[\hbox to 56mm{Feature generation}]{Математический метод предварительной дифференциальной
диагностики респираторных вирусных инфекций}
\author[N.\,P.~Ivkin]{Полина Кривуля}
\institute{Московский государственный университет имени М.В.Ломоносова}
\date{\footnotesize
\par\smallskip\emph{Курс:} Моя первая научная статья %821, 813
\par\smallskip\emph{Научный руководитель:} д.ф.-м.н. О.В.Сенько

\par\bigskip\small 2023}

%----------------------------------------------------------------------------------------------------------
\begin{document}
%----------------------------------------------------------------------------------------------------------
\begin{frame}
\thispagestyle{empty}
 \maketitle
 \end{frame}
%-----------------------------------------------------------------------------------------------------
% \begin{frame}{Goal of research}
%..
%\end{frame}
%-----------------------------------------------------------------------------------------------------
\begin{frame}{One-slide talk}

% Please comment all slides but this one if you make the One-slide talk.

\begin{columns}[c]
\column{0.6\textwidth}
\includegraphics[width=1.0\textwidth]{Surname2021PresentationSample/interval.pdf}

\column{0.4\textwidth}
    На рисунке представлена визуализация доверительных интервалов, вычисленных при $\alpha = 0.01$
\end{columns}

\bigskip
Гетерогенность между различными типами вирусов является статистически значимой. Стоит исключить симптом{ }"Боль в груди"{ }и учитывать сезон. Классификатор должен учитывать пропуски в данных.
\end{frame}


%----------------------------------------------------------------------------------------------------------
% \begin{frame}{Problem statement}
% ..
% \end{frame}
%----------------------------------------------------------------------------------------------------------
% \begin{frame}{Solution}
% \begin{columns}[c]
% \column{0.6\textwidth}
%     Column 1
% \column{0.4\textwidth}
%    Column 2
% \end{columns}
% \end{frame}
%----------------------------------------------------------------------------------------------------------
% \begin{frame}{Computational experiment}
% ..
% \end{frame}
%----------------------------------------------------------------------------------------------------------
% \begin{frame}{Conclusion}
%    \begin{block}{Forecast with hierarchical aggregation of}
%    \begin{itemize}
%        \item types of freight in
%        \item stations, regions, and roads,
%        \item for a day, week, month, and quarter.
%    \end{itemize}
%    \end{block}
% \end{frame}
%----------------------------------------------------------------------------------------------------------
\end{document} 